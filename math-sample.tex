\pdfoutput=1 % ensure pdflatex
\documentclass[10pt,twoside]{mathdoc}

% passing the option `sans` will result in sans serif 
% distribution names
% \usepackage{distoperators}
\usepackage[sans]{distoperators}

% tip: using the `IEEEeqnarray` environment is much nicer
% than the `aligned` environment
\usepackage{IEEEtrantools}

% define proposition environment
\newtheorem{prop}{Proposition}
% we can define any other amsthm environments and
% benefit from the same style

% dummy text
\usepackage{lipsum}

\title{Sample document}
\keywords{keyword 1; phrase 2.} % keywords are optional
\author[1]{First Author\thanks{This is a thanks footnote.}}
\author[2]{Second Author}
\affil[1]{Department One, Institution One, Address One}
\affil[2]{Department Two, Institution Two, Address Two}
% \author{First Author}
% \author{Second Author}
% \affil{Shared Department, Shared Institution, Shared Address}
\date{}

% set running title and authors
\makeatletter
\def\runauthor{F. Author and S. Author}
\def\runtitle{Running title}
\makeatother

\begin{document}
\maketitle
\thispagestyle{empty}
%
\begin{abstract}
    \lipsum[11]
\end{abstract}
%
%
\section{Ordinary text}
%
We reference \citet{KullbackLeibler}, before continuing our discussion.
%
\subsection{Subsections are in sans serif too}
%
\lipsum[2]
%
\subsubsection{Even subsubsections are taken care of!}
%
\lipsum[3]
%
\paragraph{And paragraphs!}
%
\lipsum[11]
%
\section{Mathematics}
%
We interest ourselves in the variables
%
\begin{IEEEeqnarray}{rl}
    X\;&\sim\normal(100, 0.5) \nonumber \\
    Y\;&\sim\Cauchy(0, 2.5). \nonumber
\end{IEEEeqnarray}
%
Note how much prettier these look.\footnote{They look good to me}

We have also defined some handy mathematics operators. For example, perhaps the following is true
%
\begin{equation}
    \E{X} = \E[x\sim \mathbb{P}_n]{x}?\label{eq:my-eq}
\end{equation}
%
Now let's get serious with our first proposition.
%
\begin{prop}
    Equation~\ref{eq:my-eq} is almost always correct under some assumptions.
\end{prop}

If we would rather not prove this, then we can talk instead about \elpdPlain, or even \elpdHatPlain if we're feeling fancy. Other mathematics functions can be found in \texttt{distoperators.sty}.
%
\begin{appendices}
%
\section{An example appendix}
%
\lipsum[4-12]
%
\end{appendices}
%
\subsubsection*{Acknowledgments}
We acknowledge all kinds of things.
%
\bibliography{sample}
%
\end{document}
