\pdfoutput=1 % ensure pdflatex
\documentclass[11pt]{swissdoc}

% passing the option `sans` will result in sans serif 
% distribution names
% \usepackage{distoperators}
\usepackage[sans]{distoperators}

% for ragged right
\usepackage{ragged2e}

% tip: using the `IEEEeqnarray` environment is much nicer
% than the `aligned` environment
\usepackage{IEEEtrantools}

% custom proposition environment
\newtheoremstyle{mystyle}%                % Name
  {}%                                     % Space above
  {}%                                     % Space below
  {\itshape}%                             % Body font
  {}%                                     % Indent amount
  {\bfseries\sffamily}%                   % Theorem head font
  {.}%                                    % Punctuation after theorem head
  { }%                                    % Space after theorem head, ' ', or \newline
  {}%                                     % Theorem head spec (can be left empty, meaning `normal')
\theoremstyle{mystyle}
\newtheorem{prop}{Proposition}
% we can define any other amsthm environments and
% benefit from the same style

% dummy text
\usepackage{lipsum}
\usepackage{multicol}

\title{Sample document}
% \title{This is a much longer title name than necessary but will be useful to showcase the title block}
\author[1]{First Author\thanks{This is a thanks footnote.}}
\author[2]{Second Author}
\affil[1]{Department One, Institution One, Address One}
\affil[2]{Department Two, Institution Two, Address Two}
\date{}

% keywords
\keywords{first topic; another keyword; and a third.}

% set running title and authors
\makeatletter
\def\runauthor{F. Author et al.}
\def\runtitle{Running title}
\makeatother

\begin{document}
\RaggedRight
%
\maketitle
%
\begin{swabstract}
    \lipsum[11]\lipsum[12]
\end{swabstract}
%
\section{Ordinary text}
%
We reference \citet{KullbackLeibler}, before continuing our discussion. \lipsum[1-3]
%
\subsection{Subsections are in sans serif too}
%
\lipsum[2]
%
\paragraph{Even paragraphs are in sans serif}
%
\lipsum[3]
%
\section{Mathematics}
%
We interest ourselves in the variables
%
\begin{IEEEeqnarray}{rl}
    X\;&\sim\normal(100, 0.5) \nonumber \\
    Y\;&\sim\Cauchy(0, 2.5). \nonumber
\end{IEEEeqnarray}
%
Note how much prettier these look\footnote{They look good to me}

We have also defined some handy mathematics operators. For example, perhaps the following is true
%
\begin{equation}
    \E{X} = \E[x\sim \mathbb{P}_n]{x}?\label{eq:my-eq}
\end{equation}
%
Now let's get serious with our first proposition.
%
\begin{prop}
    Equation~\ref{eq:my-eq} is almost always correct under some assumptions.
\end{prop}
%
\begin{proof}
    This here is my proof.
\end{proof}

If we would rather not prove this, then we can talk instead about \elpdPlain, or even \elpdHatPlain if we're feeling fancy. Other mathematics functions can be found in \texttt{distoperators.sty}.
%
\section{Buffer text and its advantages}
%
\lipsum
%
\subsection*{Acknowledgments}
We acknowledge all kinds of things.
%
\bibliography{sample}
%
\end{document}
