\pdfoutput=1 % ensure pdflatex
% \documentclass[11pt, twoside]{procdoc}
\documentclass[11pt, twoside]{procdoc2}

% tip: using the `IEEEeqnarray` environment is much nicer
% than the `aligned` environment
\usepackage{IEEEtrantools}

% for the appendix
\usepackage[title]{appendix}

% dummy text
\usepackage{lipsum}

\title{Sample document}
% \title{This is a much longer title name than necessary but will be useful to showcase the title block}
\author[1]{First Author\thanks{This is a thanks footnote.}}
\author[2]{Second Author}
\affil[1]{Department One, Institution One, Address One}
\affil[2]{Department Two, Institution Two, Address Two}
\date{}

% set running title and authors
\makeatletter
\def\runauthor{F. Author and S. Author}
\def\runtitle{Running title}
\makeatother

\begin{document}
\thispagestyle{empty}
%
\twocolumn[
  \maketitle
  \begin{onecolabstract}
    This is where the abstract goes. Be sure to copy and paste this structure when using this format.
  \end{onecolabstract}
  %
  \begin{keywords} %% Optional
    keyword 1; phrase 2; keyword 3.
  \end{keywords}
]
%
\section{Ordinary text}
%
\lipsum[1]

We reference \citet{KullbackLeibler}, before continuing our discussion: \lipsum[2]

We are also interested in maths, such as Equation~\ref{eq:1}.
%
\begin{equation}
    x = y. \label{eq:1}
\end{equation}
%
\lipsum[3]
%
Check out the wide Equation~\ref{eq:wideeq}!
%
\begin{wideeqn}
\mathcal{R}^{(\text{d})}=
 g_{\sigma_2}^e
 \left(
   \frac{[\Gamma^Z(3,21)]_{\sigma_1}}{Q_{12}^2-M_W^2}
  +\frac{[\Gamma^Z(13,2)]_{\sigma_1}}{Q_{13}^2-M_W^2}
 \right)
 + x_WQ_e
 \left(
   \frac{[\Gamma^\gamma(3,21)]_{\sigma_1}}{Q_{12}^2-M_W^2}
  +\frac{[\Gamma^\gamma(13,2)]_{\sigma_1}}{Q_{13}^2-M_W^2}
 \right)\;. \label{eq:wideeq}
\end{wideeqn}

\lipsum[4-8]
%
\subsection*{Acknowledgments}
We acknowledge all kinds of things.
%
\bibliography{sample}
%
% The appendix to the paper can be made one column
%
\onecolumn
\appendix
%
\section{Appendix}
%
\lipsum[12-15]
%
\end{document}
